\documentclass{article}
\usepackage{amsmath}
\usepackage[english,russian]{babel}
\usepackage[left=3cm,right=3cm]{geometry}
\usepackage[utf8]{inputenc}
\usepackage[T2A]{fontenc} 
\usepackage{indentfirst}
\usepackage{algorithm}
\usepackage{algpseudocode}
\title{Оптимизация светофорного регулирования для зелёной волны}
\author{}
\date{}
\begin{document}
\maketitle
\thispagestyle{empty}
\newpage
\tableofcontents
\newpage
\section{Постановка задачи}
Настроить работу четырёх последовательных светофоров (tls#0–tls#3) таким образом, чтобы обеспечить бесперебойное движение транспортных средств по всему участку. Основная наша задача состоит в подборе:
\begin{enumerate}
\item Оптимальных сдвигов циклов светофоров относительно друг друга.
\item Длительности сигналов (зелёный, жёлтый, красный) в рамках заданного цикла длительностью 85 секунд.
\end{enumerate}
\par Первостепенной задачей является создание условий, при которых автомобиль, начавший движение на зелёный сигнал первого светофора (tls#0), проезжает весь маршрут до конечного светофора (tls#3) без остановок. Это требует следующих согласований:
\begin{enumerate}
\item Времени начала циклов (сдвигов) между светофорами.
\item Длительности разрешающих (зелёных) и запрещающих сигналов с учётом скорости движения и расстояний между перекрёстками.
\end{enumerate}
\par После достижения бесперебойности необходимо максимизировать пропускную способность участка за счёт оптимизации длительности сигналов в рамках 85-секундного цикла.
\newpage
\section{Алгорим оптимизации светофорного регулирования для зеленой волны}    
\begin{algorithm}
\begin{algorithmic}[1]
\State \textbf{Инициализация:}
\State $V \gets 50$ \Comment{Скорость (км/ч)}
\State $T_{\text{цикл}} \gets 85$ \Comment{Длительность цикла (сек)}
\State $S \gets [0,200,450,600]$ \Comment{Координаты светофоров (м)}

\State \textbf{Фазы светофоров:}
\State Для светофора 0: фаза 1 (Зеленый:30сек, Красный:20сек), фаза 2 (Зеленый:15сек, Красный:15сек)
\State Для светофора 1: фаза 10 (Красный:20сек, Зеленый:35сек, Ж:5сек), фаза 11 (Красный:10сек, Зеленый:10сек, Желтый:5сек)
\State Для светофора 2: фаза 20 (Красный:45сек,Зеленый:10сек), фаза 21 (Красный:7сек, Зеленый:18сек, Желтый:5сек)
\State Для светофора 3: фаза 30 (Красный:40с, Зеленый:15с), фаза 31 (Красный:10с, Зеленый:20с)

\Function{проезд}{$T_{\text{смещение}}$}
    \For{$i \gets 1$ \textbf{до} $4$}
        \State $t_{\text{прибытие}} \gets \frac{S[i]}{V \cdot 1000 / 3600}$
        \State $t_{\text{сигнал}} \gets (t_{\text{прибытие}} + T_{\text{смещение}) \bmod T_{\text{цикл}}$
        \If{$\text{Сигнал}(i, t_{\text{сигнал}}) = \text{Красный}$}
            \State $T_{\text{стоп}} \gets \text{Длительность красного}(i, t_{\text{сигнал}})$
            \State $T_{\text{смещение}} \gets T_{\text{смещение}} + T_{\text{стоп}}$
            \State \Return \textbf{false}
        \EndIf
    \EndFor
    \State \Return \textbf{true}
\EndFunction

\State \textbf{Оптимизация:}
\While{$\text{Проезд}(T_{\text{смещение}}) = \textbf{false}$}
    \State $T_{\text{смещение}} \gets T_{\text{смещение}} + 1$
    \If{$T_{\text{смещение}} > T_{\text{цикл}}$}
        \State $V \gets V + 5$
        \State $T_{\text{смещение}} \gets 0$
    \EndIf
\EndWhile

\State \textbf{Результат:}
\State Оптимальный сдвиг: $T_{\text{смещение}}$ (сек)
\State Лучшая скорость: $V$ (км/ч)

\end{algorithmic}
\end{algorithm}

\newpage
\section{Язык программирования и обоснование выбора}
Для реализации проекта мной был выбран язык программирования Python. Этот выбор обусловлен его соответствием требованиям задачи и имеющимся опытом работы с данным языком. Его синтаксис позволяет реализовывать сложные алгоритмы, такие как в данной задаче. Вместо того, чтобы тратить время на борьбу с изучением нового языка, можно сосредоточиться на самом сложном— реализации.  
\par В Python есть большой выбор полезных библиотек. Например, NumPy, SciPy, Matplotlib, tkinter. Эти библиотеки могут помочь с численными вычислениями, оптимизацией и минимизацией функций, построением графиков, а также графическим интерфейсом(если будет такая необходимость). 
\par Кроме того, Python отлично подходит для научных и инженерных задач. Если необходимо будет расширить проект — например, добавить анализ реальных данных или подключить SQL — Python с этим справится без проблем. 
\par Использование Python также сокращает затраты на освоение инструментов и настройку среды для программирования. Именно поэтому Python оказался тем самым инструментом, который соответствует и требованиям проекта, и моим навыкам.



\end{document}